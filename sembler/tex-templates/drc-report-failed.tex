\documentclass[twoside]{article}
\usepackage{datetime2,hyperref,color}
% %% Style file for the sembler process

\NeedsTeXFormat{LaTeX2e}
\ProvidesPackage{sembler}
\RequirePackage{color}
\RequirePackage{graphicx}
\RequirePackage{hyperref}
%% Set up our colors to match the style guide
\definecolor{coolBlack}{rgb}  {0.157, 0.247 , 0.302}
\definecolor{draperRed}{rgb}  {1    , 0.275 , 0.071}
\definecolor{semblerBlue}{rgb}{0.255, 0.671 , 0.878}
\definecolor{printBlack}{rgb} {0.067, 0.094 , 0.125}
%% Note white isn't in spec

%% Set up the fonts
\renewcommand{\rmdefault}{gothic}
\renewcommand{\sfdefault}{gothic}
\renewcommand{\ttdefault}{gothic}

\newcommand \sect[1]{\section{\textcolor{draperRed}{#1}}}
\newcommand \subsect[1]{\subsection{\textcolor{semblerBlue}{#1}}}
\newcommand \puritysect[1]{\subsection{\textcolor{draperRed}{#1}}}
\newcommand \rulesect[1]{\subsection{\textcolor{coolBlack}{#1}}}
\newcommand \suggsect[1]{\subsection{\textcolor{semblerBlue}{#1}}}
\newcommand \addpage[1]{\begin{center}\includegraphics[width=\textwidth,
      height=\textheight, keepaspectratio]{#1}\end{center}}

\title{Sembler DRC Report Content}
\author{Nivair H. Gabriel}
\date{\today}

\begin{document}

% \maketitle

\section{Summary}
\begin{tabular}{lr}
Design file:& $<$FILENAME$>$.dxf\\
Design status:& FAILED\\
Design rule check performed on:& \DTMnow\\
Design Rule Checker (DRC) version:& $<$VERSION$>$\\
\end{tabular}

\section{Output}
The DRC generated a ZIP file with feedback on your design. This ZIP file
contains:
\begin{itemize}
\item \texttt{$<$FILENAME$>$.pdf}: this report
\item \texttt{$<$FILENAME$>$.dxf}: your submitted design file
\item \texttt{$<$FILENAME$>$-output.dxf}: an ``output DXF'' file that contains your design with notation from the DRC
\end{itemize}
This report guides you in interpreting the DRC feedback.

\section{Design status}
\subsection*{FAILED}
Your design has features that violate one or more of the Design Rules (see
\href{https://sembler.draper.com/files/sembler_process_rules.pdf}{Sembler Design
  Rules and Process Description (PDF)}). In order for Sembler to fabricate your
design, it needs to pass the Design Rule Check with no rule violations.

For more information, review the reported errors and the checkplots in this
document, as well as the output DXF file. When you have fixed these errors,
resubmit your updated design. If you require assistance, contact
\href{mailto:sembler@draper.com}{sembler@draper.com}.

\section{Errors and Warnings}
\subsection{Issues per location}
\begin{tabular}{l|c|c}
 & \# Errors & \# Warnings \\
 & (Design Rule violations)& (Design Guidelines exceeded) \\ \hline
In multiple layers or overall design& 6& 0\\
In L0\_METAL& 1& 0\\
In L1\_SU8 & 1& 0\\
In L2\_SU8 & 2& 0\\
In L3\_SU8 & 2& 0\\
\end{tabular}

\subsection{General design issues}

\subsubsection{Die boundary too small}
\par The die boundary is smaller than the minimum die size. The die must be at
least 1 cm x 1 cm (10000 $\mu$m x 10000 $\mu$m).  \par See
ERR\_MIN\_WIDTH\_FEATURE\_DIE\_BOUNDARY layer in output DXF file.

\subsubsection{Die boundary too large}
\par The die boundary is larger than the maximum die size. A rectangular die has
dimensions ranging in integral sizes from 1 cm x 1 cm (10000 $\mu$m x 10000
$\mu$m) to 4 cm x 4 cm (40000 $\mu$m x 40000 $\mu$m), or microscope slide
dimensions of 2.5 cm x 7.5 cm (25000 $\mu$m x 75000 $\mu$m).  \par See
ERR\_MAX\_WIDTH\_FEATURE\_DIE\_BOUNDARY layer in output DXF file.

\subsubsection{Die boundary not rectangular}
\par The die boundary must be rectangular. Dies are available in rectangular
shapes with integral dimensions ranging from 1 cm x 1 cm (10000 $\mu$m x 10000
$\mu$m) to 4 cm x 4 cm (40000 $\mu$m x 40000 $\mu$m), or with microscope slide
dimensions of 2.5 cm x 7.5 cm (25000 $\mu$m x 75000 $\mu$m).  \par See
ERR\_SHAPE\_DIE\_BOUNDARY layer in output DXF file.

\subsubsection{Die boundary not in standard size}
\par The die boundary must be the size of one of Sembler's standard dies. Dies
are available in rectangular shapes with integral dimensions ranging from 1 cm x
1 cm (10000 $\mu$m x 10000 $\mu$m) to 4 cm x 4 cm (40000 $\mu$m x 40000 $\mu$m),
or with microscope slide dimensions of 2.5 cm x 7.5 cm (25000 $\mu$m x 75000
$\mu$m).  \par See ERR\_SIZE\_DIE\_BOUNDARY layer in output DXF file.

\subsubsection{Too few alignment marks}
\par Three alignment marks in proper configuration must be present. To see
predefined alignment marks in proper configuration (triangular pattern with no
more than two co-linear), download the Sembler template files and open a blank
design template.  \par See ERR\_NUMBER\_ALIGNMENT\_MARK layer in output DXF
file.

\subsubsection{Alignment marks not in proper configuration}
\par Three alignment marks in proper configuration must be present. To see
predefined alignment marks in proper configuration (triangular pattern with no
more than two co-linear), download the Sembler template files and open a blank
design template.  \par See ERR\_ARRANGEMENT\_ALIGNMENT\_MARK layer in output DXF
file.

\subsection{Issues in L0\_METAL}

\subsubsection{Feature outside die boundary}
\par A feature in L0\_METAL is outside the die boundary. The die boundary must
enclose all metal features.  \par See ERR\_ENCLOSE\_L0\_METAL\_IN\_DIE\_BOUNDARY
layer in output DXF file.

\subsection{Issues in L1\_SU8}

\subsubsection{Feature outside die boundary}
\par A feature in L1\_SU8 is outside the die boundary. The die boundary must
enclose all SU-8 features.  \par See ERR\_ENCLOSE\_L1\_SU8\_IN\_DIE\_BOUNDARY
layer in output DXF file.

\subsection{Issues in L2\_SU8}

\subsubsection{Feature outside die boundary}
\par A feature in L2\_SU8 is outside the die boundary. The die boundary must
enclose all SU-8 features.  \par See ERR\_ENCLOSE\_L2\_SU8\_IN\_DIE\_BOUNDARY
layer in output DXF file.

\subsubsection{Unsupported L2\_SU8 feature}
\par A feature in L2\_SU8 is not supported by L1\_SU8 below it. All L2\_SU8
features must have L1\_SU8 features defined beneath them.  \par See
ERR\_ENCLOSE\_ALL\_L2\_SU8\_IN\_L1\_SU8 layer in output DXF file.

\subsection{Issues in L3\_SU8}

\subsubsection{Feature outside die boundary}
\par A feature in L3\_SU8 is outside the die boundary. The die boundary must
enclose all SU-8 features.  \par See ERR\_ENCLOSE\_L3\_SU8\_IN\_DIE\_BOUNDARY
layer in output DXF file.

\subsubsection{Unsupported L3\_SU8 feature}
\par A feature in L3\_SU8 is not supported by L2\_SU8 below it. All L3\_SU8
features must have L2\_SU8 features defined beneath them.  \par See
ERR\_ENCLOSE\_ALL\_L3\_SU8\_IN\_L2\_SU8 layer in output DXF file.

\section{Checkplots}
Review the checkplots for the SU-8 and metal layers to verify that the Design
Rule Checker correctly interprets your design.  \par Fluidic and metal features
appear in white, while substrate and bulk PDMS appear in gray. For reference,
ports and alignment marks are shown in each layer as black dashed lines. Note
also that the DRC automatically maps support posts to the SU-8 layers where
features exist.

\end{document}
